\documentclass[stu,12pt,floatsintext]{apa7}

% Language and citation setup
\usepackage[american]{babel}
\usepackage{csquotes}
\usepackage[style=apa,sortcites=true,sorting=nyt,backend=biber]{biblatex}
\DeclareLanguageMapping{american}{american-apa}
\addbibresource{references.bib}

% Font and encoding
\usepackage[T1]{fontenc}
\usepackage{newtxtext,newtxmath}  % Modern Times-like font with math support

% Document metadata
\title{Reflection on the Image of God}
\author{Lanie Molinar}
\authorsaffiliations{Colorado Christian University}
\duedate{June 30, 2025,}
\course{Introduction to Systematic Theology (THE-200A)}
\professor{Dr. Cari Nimeth}

\begin{document}

\maketitle
\thispagestyle{plain}
\pagestyle{plain}

\section{Views of the Image of God}

According to \textcite[Chapter 18]{ericksonIntroducingChristianDoctrine2015}, there are three views regarding the image of God in humanity. These views are the substantive view, the relational view, and the functional view. I will briefly outline each view and its problems. I will also compare and contrast these views to provide a comprehensive understanding of the image of God.

The substantive, relational, and functional views each offer unique insights into the nature of the image of God in humanity. The substantive view emphasizes inherent qualities and attributes, the relational view focuses on the importance of relationships, and the functional view highlights the roles and responsibilities individuals have in reflecting God's image. Each perspective contributes to a more comprehensive understanding of what it means to be made in the image of God, and together they invite a holistic approach to theology and ethics. Each view has its strengths and weaknesses, and a balanced understanding of the image of God in humanity requires consideration of all three perspectives. The substantive view risks reducing the image of God to measurable qualities, such as rationality or morality, which may overlook relational and functional aspects of humanity. This can unintentionally promote a hierarchy of human worth, where those with certain traits are seen as more fully bearing God's image, potentially excluding individuals with cognitive disabilities. The relational view, while emphasizing community and connection, may downplay individual responsibility and struggle to explain how the image of God is present in those who are isolated or unable to form relationships. The functional view focuses on human roles and responsibilities, but it may reduce the image to what people do rather than who they are, raising concerns about the dignity of those unable to fulfill certain functions due to age, disability, or other limitations.

\section{Erickson’s Conclusions}

After examining the limitations of the substantive, relational, and functional views, \textcite{ericksonIntroducingChristianDoctrine2015} concludes that the image of God cannot be fully defined by any one perspective. Instead, he draws several inferences from Scripture that emphasize the image as something universal, enduring, and intrinsic to human nature. Rather than being based on what humans do or how they relate, the image is primarily about what humans are, beings created with the capacity for personality, reflection, and relationship. Erickson concludes that the image of God is universal, enduring, and equally present in all people. It is not dependent on intelligence, ability, or social status, and cannot be lost through sin. Rather than being defined by function or relationship, the image is structural, part of what humans are. It includes the qualities that enable humans to fulfill their purpose, such as personality, reflection, and moral agency.

These conclusions help us understand that being human is not defined by our abilities, achievements, circumstances, or social roles, but by our very nature as beings created in the image of God. Erickson emphasizes that the image of God is something every person possesses permanently and to an equal degree, regardless of their situation or condition. This understanding affirms the inherent dignity and worth of all individuals, calling us to recognize and honor the divine image in ourselves and others. It reminds us that our capacity for relationship, creativity, reflection, and moral decision-making is central to who we are as humans and reflects the image of God within us. It also encourages us to live in a way that reflects God's character, especially through love, service, obedience, and fellowship, as modeled by Jesus Christ.

\section{Personal Reflection}

Erickson's third implication of the doctrine of the image of God is that "We experience full humanity only when we are properly related to God" \parencite{ericksonIntroducingChristianDoctrine2015}. I think this is the most meaningful implication to me because I am growing closer to God, so this means I am \textit{becoming more human in the truest sense}. This means a lot to me as a woman with multiple disabilities, living in a world that often makes me feel less than human, a world that seems to think it is not worth making things accessible, especially for those with multiple disabilities. Understanding that my relationship with God is what truly defines my humanity helps me to embrace my identity and worth, regardless of societal perceptions or limitations. I am learning to see myself through God's eyes, recognizing the inherent value and dignity that comes from being created in His image.

\newpage

\section{References}

\printbibliography

\end{document}